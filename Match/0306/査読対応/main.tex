\documentclass[12pt,dvipdfmx]{report}
\usepackage[dvips]{graphicx}
\usepackage{amsmath,amssymb,amsfonts}
\usepackage{color}
\usepackage{ics}
\usepackage{latexsym}
\usepackage{makeidx}
\usepackage{moreverb}
\usepackage{tabularx}
\usepackage{verbatim}

%%%%% 修論特有ページ設定(変更不要) %%%%%%%%%%%%%%%%%%%%%%%%%%%%%%%%%%%%%%%%%%%%%%%%%%%%

\setcounter{topnumber}{5}%    ページ上部の図表は 5 個まで
\def\topfraction{1.00}%       ページの上 1.00 まで図表で占めて可
\setcounter{bottomnumber}{5}% ページ下部の図表は 5 個まで
\def\bottomfraction{1.00}%    ページの下 1.00 まで図表で占めて可
\setcounter{totalnumber}{10}% ページあたりの図表は 10 個まで
\def\textfraction{0.04}%      ページうち本文が占める割合の下限

\def\epsfsize#1#2{\ifnum#1>\hsize\hsize\else#1\fi}
%\setlength{\baselineskip}{1.5\baselineskip} %double space

%%%%% 修論特有コマンド定義(変更不要) %%%%%%%%%%%%%%%%%%%%%%%%%%%%%%%%%%%%%%%%%%%%%%%%%%%%

\newtheorem{definition}{定義}[chapter]
\newtheorem{algorithm}{アルゴリズム}[chapter]

%%%%% 貢献への色付け(変更不要) %%%%%%%%%%%%%%%%%%%%%%%%%%%%%%%%%%%%%%%%%%%%%%%%%%%%

\newif\ifdraft
\drafttrue %貢献の色を付ける場合
%\draftfalse %貢献の色を消す場合

\ifdraft
\newcommand{\controne}[1]{\color{red} #1 \color{black}}
\newcommand{\contrtwo}[1]{\color{green} #1 \color{black}}
\newcommand{\contrthree}[1]{\color{blue} #1 \color{black}}
\else
\newcommand{\controne}[1]{#1}
\newcommand{\contrtwo}[1]{#1}
\newcommand{\contrthree}[1]{#1}
\fi

%%%%% 目次生成(変更不要) %%%%%%%%%%%%%%%%%%%%%%%%%%%%%%%%%%%%%%%%%%%%%%%%%%%%

\makeindex

\begin{document}

%%%%% タイトル定義 %%%%%%%%%%%%%%%%%%%%%%%%%%%%%%%%%%%%%%%%%%%%%%%%%%%%

\papercode{ICS-***-***}
\title{English Title}
\jtitle{日本語タイトル}
\major{工学部情報工学科}
\author{著者名}
\date{令和Y年MM月DD日提出}
\supervisor{安積 卓也 教授}
\labname{安積研究室}
\studentID{XXTIYYY}
\labaffiliation{埼玉大学 理工学研究科・工学部} % 変更不要
\maketitle

%%%%% Abstract %%%%%%%%%%%%%%%%%%%%%%%%%%%%%%%%%%%%%%%%%%%%%%%%%%%%

\pagenumbering{roman}
\chapter*{Abstract}
\addcontentsline{toc}{chapter}{Abstract}
Here is abstract.

%%%%% Contents(変更不要) %%%%%%%%%%%%%%%%%%%%%%%%%%%%%%%%%%%%%%%%%%%%%%%%%%%%

\tableofcontents

\listoffigures
\addcontentsline{toc}{chapter}{List of Figures}

\listoftables
\addcontentsline{toc}{chapter}{List of Tables}

%%%%% Introduction %%%%%%%%%%%%%%%%%%%%%%%%%%%%%%%%%%%%%%%%%%%%%%%%%%%%

\chapter{Introduction}
\pagenumbering{arabic}
\setcounter{page}{1}


\section{Background}

Background

\section{Purpose and Objective}

Purpose and Objective

The contributions of this thesis are as follows:
\begin{itemize}
    \item \controne{Contribution one.}
    \item \contrtwo{Contribution two.}
    \item \contrthree{Contribution three.}
\end{itemize}

\section{Construction}
The rest of thesis organised as follows:
Chapter Y explains ... 
Chapter Z gives conclusion and future works.



\chapter{Figure Sample}

Figure~\ref{sample_figure}.

\begin{figure}[tbp]
    \centering
    \includegraphics[width=\linewidth]{fig/sample.eps}
    \caption{Sample of figure.}
    \label{sample_figure}
\end{figure}



\chapter{Table Sample}

Table~\ref{sample_table}.

\newcolumntype{Y}{>{\centering\arraybackslash}X}
\begin{table}[tbp]
    \centering
    \caption{Comparison of this thesis with existing studies.}
    \begin{tabularx}{\linewidth}{|l|Y|Y|Y|Y|}\hline
                                        &    Hoge    &    Fuga    &    Piyo    &    Neko     \\ \hline
    A method~\cite{kato2018autoware}    &            & \checkmark & \checkmark & \checkmark  \\ \hline
    B method~\cite{li2022autoware_perf} & \checkmark &            & \checkmark & \checkmark  \\ \hline
    C method~\cite{dds1.4}              & \checkmark & \checkmark &            & \checkmark  \\ \hline
    D method                            & \checkmark &            & \checkmark &             \\ \hline
    This thesis                         & \checkmark & \checkmark & \checkmark & \checkmark  \\ \hline
    \end{tabularx}
    \label{sample_table}
\end{table}

\chapter{Fugafuga}
\section{Foofoo}
hehehehe

\section{Hyohyo}

fufufufufufufu.

\chapter{Conclusion}
\section{Summary}

We have ...

\section{Future Works}

Future works are as follows: ...

\newpage

%%%%% Acknowledgements %%%%%%%%%%%%%%%%%%%%%%%%%%%%%%%%%%%%%%%%%%%%%%%%%%%%
\chapter*{Acknowledgements}
\addcontentsline{toc}{chapter}{Acknowledgements}
\noindent
Here is acknowledgements.

\newpage

%%%%% Publications %%%%%%%%%%%%%%%%%%%%%%%%%%%%%%%%%%%%%%%%%%%%%%%%%%%%

\chapter*{Publications}
\addcontentsline{toc}{chapter}{Publications}

\begin{list}%
 {} %default label
 {} %formatting parameter
 \item Refereed papers published in journals or books (first author)
       \begin{itemize}
	\item Hogehoge: ....
       \end{itemize}
 \item Refereed papers published in journals or books (co-author)
       \begin{itemize}
	\item Hogehoge: ....
       \end{itemize}
 \item Refereed papers published in international conference proceedings (first author)       \begin{itemize}
	\item Hogehoge: ....
       \end{itemize}

 \item Refereed papers published in international conference proceedings (co-author)
       \begin{itemize}
	\item Hogehoge: ....
       \end{itemize}
 \item Refereed papers published in local conference proceedings (first author)
       \begin{itemize}
	\item Hogehoge: ....
       \end{itemize}
 \item Refereed papers published in local conference proceedings (co-author)
       \begin{itemize}
	\item Hogehoge: ....
       \end{itemize}
\end{list}

%%%%% Reference(変更不要) %%%%%%%%%%%%%%%%%%%%%%%%%%%%%%%%%%%%%%%%%%%%%%%%%%%%

\bibliographystyle{unsrt}
\bibliography{reference}

%%%%% Appendix %%%%%%%%%%%%%%%%%%%%%%%%%%%%%%%%%%%%%%%%%%%%%%%%%%%%

\renewcommand{\thesection}{\Alph{section}}
\appendix
\addcontentsline{toc}{chapter}{Appendix}
\chapter{Huroku}

kokokokoo.

\chapter{MataHuroku}

kokokokoko.

%%%%% Appendix ここまで %%%%%%%%%%%%%%%%%%%%%%%%%%%%%%%%%%%%%%%%%%%%%%%%%%%%

%\printindex
%\addcontentsline{toc}{chapter}{Index}
%目次にIndexを表示させる場合は.idxファイルのtheindex環境の中で
%上記コマンドを置く
\end{document}
